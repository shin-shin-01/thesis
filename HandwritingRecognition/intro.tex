%#! platex thesis.tex

%======================================================================
\chapter{はじめに}
\label{cha:intro}
\section{研究背景}
\label{sec:background}
Portable Health Clinic (以下,PHC)は発展途上国農村部における,健康促進のための遠隔医療システムである\cite{ahmed15:portable}.ヘルスアシスタントと呼ばれるスタッフが複数の健康測定器具を医者のいない農村部に持ち込み,村民に対して健康診断を行う.健康診断の結果,医者からの診断が必要であると判断された患者は都市部にいる医者と電話を通して繋がり,診断を受けることができる.このシステムによって,医者が直接診断できない発展途上国農村部においても人々は診断を受けることができる.このシステムでは,医者は患者を診断しながら症状や処方薬などをノートに取り,通話後にそれをコンピュータに入力して処方箋を作成する.このときにノートに書かれた手書き文字を認識し,その情報を元に処方箋を予測してコンピュータに入力する手間を削減できれば,医者の時間を節約でき,医者はさらに多くの人々の診断を行うことができる.

\section{研究目的}
ここを大きく変更(機械学習モデルについての考察はあまりやっていないのでそこをどうするか考える必要がある)

本研究の目的は2つある.1つ目は遠隔医療における,処方箋予測に向けた手書き文字認識に適している機械学習モデルについて考察を行うことである.現在,医者の手書き文字認識について言及している研究は筆者の調査の範囲内には存在しない.医者の手書き文字には筆記体・医療用語・複数言語などの様々な問題があり,既存の技術のうちどのようなものを用いるべきかについて議論がなされるべきである.

2つ目は機械学習を行うにあたって,オンライン文字認識におけるデータの拡張方法を確立することである.オープンソースのオンライン文字認識用データセットのうち,医療用語に特化したものは存在していないため,本研究では独自にデータ収集を行う必要がある.しかしデータの収集には多くの時間と労力を要する.そこで本研究ではデータ拡張を行ってデータ量を水増しする.現在,オンライン文字認識の研究においてデータ拡張を行っているものは非常に少なく,オンライン文字認識の拡張方法は未だ確立されていない.機械学習においてデータ量は精度を大きく左右する重要なパラメータである.オフライン文字認識については多くのデータ拡張方法が研究されているが,オンライン文字認識についてもデータの拡張方法について議論がなされるべきである.

\section{論文構成}
本論文の構成は以下の通りである.第2章では,手書き文字認識における関連技術を説明し,オンライン手書き文字認識とデータ拡張についての関連研究を説明する.第3章では提案手法の概要と,提案手法における各要素について詳しく述べる.第4章では提案手法の実装について述べる.第5章では提案手法の評価について述べる.最後に第6章で本論文のまとめと今後の展望を述べる.
