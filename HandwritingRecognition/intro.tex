%#! platex thesis.tex

%======================================================================
\chapter{はじめに}
\label{cha:intro}
\section{研究背景}
\label{sec:background}
Portable Health Clinic (以下,PHC)は発展途上国農村部における,健康促進のための遠隔医療システムである\cite{ahmed15:portable}.ヘルスアシスタントと呼ばれるスタッフが複数の健康測定器具を医者のいない農村部に持ち込み,村民に対して健康診断を行う.健康診断の結果,医者からの診断が必要であると判断された患者は都市部にいる医者と電話を通して繋がり,診断を受けることができる.このシステムによって,医者が直接診断できない発展途上国農村部においても人々は診断を受けることができる.このシステムでは,医者は患者を診断しながら症状や処方薬などをノートに取り,通話後にそれをコンピュータに入力して処方箋を作成する.このときにノートに書かれた手書き文字を認識し,その情報を元に処方箋を予測してコンピュータに入力する手間を削減できれば,医者の時間を節約でき,医者はさらに多くの人々の診断を行うことができる.

\section{研究目的}
本研究の目的は,遠隔医療における,処方箋予測に向けたオンライン文字認識に適したデータの拡張方法を確立することである.オープンソースのオンライン文字認識用データセットのうち,医療用語に特化したものは存在していないため,先行研究\cite{takahashi}では独自にデータ収集を行っている.しかしデータの収集には多くの時間と労力を要するため,先行研究\cite{takahashi}ではストロークの回転と平行移動によっ てデータ量を水増しする SRP(Stroke Rotation and Parallel-shift)手法を行ってデータ量を水増している.しかし,SRP手法において回転の大きさ,平行移動の大きさは筆者が文字の形を崩さないと判断できる範囲で設定しており,データを高倍率で拡張した場合同じようなパラメーターが増えてしまい,拡張後に同じようなデータが増えて学習精度が低くなる原因になってしまう.そのため,この用語認識は学習の際にランダムに設定された回転の大きさと平行移動の大きさの組み合わせにより,認識精度にばらつきが出てしまうという問題がある.

現在,オンライン文字認識の研究においてデータ拡張を行っているものは非常に少なく,安定して高精度で認識できるオンライン文字認識の拡張方法は未だ確立されていない.機械学習においてデータ量は精度を大きく左右する重要なパラメータである.オフライン文字認識については多くのデータ拡張方法が研究されているが,オンライン文字認識についてもデータの拡張方法について議論がなされるべきである.


\section{論文構成}
本論文の構成は以下の通りである.第2章では,筆者らが目指しているオンライン手書き医療用語認識,及び本研究に至った取り組むべき課題について説明を行い,既存の対策の問題点をあげる.第3章では,問題解決に向け本研究で提案する手法について説明を行う.第4章では提案手法の実装について述べる.第5章では提案手法の評価について述べる.最後に第6章で本論文のまとめと今後の展望を述べる.
