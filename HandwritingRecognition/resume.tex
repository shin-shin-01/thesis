%#! platex resume.tex
\documentclass[a4j,11pt]{jreport}
\usepackage{resume}

%======================================================================
% タイトル
\title{手書き医療用語認識における\\データ拡張手法の検討}
\author{新堂風}

% 修士は以下のコメントを有効化する
%\subtitle{修士論文試問予稿}
%\university{九州大学大学院}
%\department{システム情報科学府}
%\major{情報知能工学専攻}

\professor{久住 憲嗣 准教授}
\date{令和2年2月14日}
\time{13:10〜13:25}
\location{システム情報科学第302講義室}

%======================================================================
\begin{document}
\maketitle

Portable Health Clinic(以下,PHC) は発展途上国農村部における健康促進に向けた遠隔医療システムである.このシステムでは都市部の医者は遠隔地の患者を電話で診断しながら症状や処方薬などをノートに取り,通話後にそれをコンピュータに入力して処方箋を作成する.このときにノートに書かれた手書き文字を認識し,それを元に処方箋を予測してコンピュータに入力する手間を削減できれば,医者はさらに多くの人々の診断を行うことができる.

現在医療用語に特化したデータセットが存在しないため,独自に単語を座標の時系列データとして収集する.様々な文字に対応するため大量のデータを得る必要があるが,十分なデータ量を得るためには多くの労力を要する.少ないデータ量での高精度の医療用語認識を目指して,現在データ拡張手法を用いたオンライン手書き医療用語認識が開発されている.先行研究ではストロークの回転と平行移動によりデータ量を水増しするデータ拡張手法を用いている.しかし回転と平行移動の大きさは筆者が文字の形を崩さない範囲で設定しているため,高倍率で拡張した時に近い値を用いてしまい拡張後に同じようなデータが増えてしまっている.そのためこの用語認識は認識精度にばらつきが出るという問題がある.

本論文では処方箋の予測に向けたシステムの初期研究として,安定して高精度で学習できるオンライン手書き医療用語認識システムを提案する.またオンライン手書き文字のデータ拡張手法として文字の縦横比を変更しデータ量を水増しするRatio手法を新たに提案する.Ratio手法は文字の縦横比を変更するため,拡張後に文字のストロークが重なることがなく文字の形を大きく変えてしまわないままデータの多様化を行うことができる.本論文では既存の拡張手法とRatio手法を組み合わせてデータ拡張を実現した.収集した15991語のデータを用いて,既存手法とRatio手法でデータ拡張を行った後にBidirectionalLSTMで学習を行った結果,480語のクラスにおいて平均精度93.0\%,最小精度92.1\%で単語を認識した.この結果は既存手法のみの場合の認識精度と比べて平均精度は3.5\%,最小精度は14.7\%向上した.

\preseninfo
\end{document}
