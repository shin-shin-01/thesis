%#! platex thesis.tex

%======================================================================
\chapter{おわりに}
\label{cha:conclu}

%----------------------------------------------------------------------
\section{本研究の主たる成果}
\label{sec:main-result}

本研究では,遠隔医療における処方箋予測に向けた,BidirectionalLSTMを用いたオンライン手書き医療用語認識手法の提案と,オンライン文字認識用のデータ拡張手法である SRP(Stroke Rotation and Parallel-shift) 手法・ChangeRatio手法の提案を行った.独自にデータ収集を行う必要があったため,コーパスとしてPHCの過去の処方箋データを用いることで処方箋に頻出する医療単語を抽出しデータ収集を行った.また,様々な手書き文字に対応するため,ストロークの座標を移動させることによるデータ拡張(SRP手法)・ーーーーーーーーー(ChangeRatio手法)を行った.拡張したデータを入力としてBidirectionalLSTMを用いて評価を行った.

評価の結果,収集した15991語のデータと480語のクラスにおいて=====の精度で単語を認識した.この結果はデータ拡張を行わなかった場合の認識精度と比べて=====高かった.
%----------------------------------------------------------------------
\section{今後の課題}
\label{sec:future}

% 以下はRefTeX用
%%% Local Variables:
%%% mode: yatex
%%% TeX-master: "thesis"
%%% End:
