%#! platex thesis.tex

%======================================================================
\chapter{おわりに}
\label{cha:conclu}

%----------------------------------------------------------------------
\section{本研究の主たる成果}
\label{sec:main-result}

本論文では,遠隔医療における処方箋予測に向けたオンライン手書き医療用語認識における,オンライン文字認識のデータ拡張手法であるRatio手法の提案を行った.まず最初に先行研究でのデータ拡張では,拡張後に同じようなデータが増えてしまい最小精度が低くなってしまうと指摘した.この問題を解決するためにRatio手法の提案とSRP手法と組み合わせた際の精度を安定させる拡張倍率の分析を行った.
新手法Ratioを組み合わせたデータ拡張手法の精度評価の結果,収集した15991語のデータと480語のクラスにおいて平均精度 93.0\%,最小精度 90.4\%で単語を認識した.の結果 は SRP 手法のみの場合の認識精度と比べて平均精度は 3.5\%,最小精度は 22.4\%高かった.データ拡張を行わなかった場合の認識精度と比べて平均精 度は 19.6\%,最小精度は 71.4\%高かった
%----------------------------------------------------------------------
\section{今後の課題}
\label{sec:future}
今後の課題としては以下の3つがあげられる.1つ目は拡張手法の拡張倍率とパラメータの調整である.機械学習において似たようなデータが増えた状態で学習を行うと,学習データに対する過学習で精度が低くなってしまう.また機械学習や前処理で多くのデータを利用すると,コンピュータへの負荷や学習時間が増加してしまうため学習に良い影響を及ぼさないデータは余計に増やすべきではない.そのため,似たようなデータが増えない最小の拡張倍数・パラメータでデータ拡張を行い精度を向上させる必要がある.本研究では最終的にデータを125倍に拡張してデータ拡張を行ったが,それぞれの拡張倍数の組み合わせや拡張時のパラメータは実験を重ねて考察する必要がある.


% 以下はRefTeX用
%%% Local Variables:
%%% mode: yatex
%%% TeX-master: "thesis"
%%% End:
