\section{はじめに}
技術的負債とは,コード中に存在するバグや解消すべき課題のことであり,その中でも開発者が課題を把握したうえで,コードに埋め込まれた技術的負債をSelf-Admitted Technical Debt(SATD)を呼ぶ.SATDを解消することは,ソフトウェアの品質向上につながることから,近年様々なアプローチで研究が実施されている.

\par

他方,近年ソフトウェアのクラウド化に伴い,コンテナ型仮想化技術のひとつである,Dockerが注目されている.Dockerでは,コンテナ型の仮想環境を作成,配布,実行することができ,軽量で高速に実行し,停止できるため,様々なプロジェクトで利用されている.
Dockerにおいても,従来のSATD研究で調査対象とされてきた一般的なプログラミング言語と同様に,SATDの存在が報告されている\cite{docker-satd}.しかし,DockerにおけるSATDの削除についての調査はまだ行われていない.
SATDの解消手法やそのパターンは,開発者を支援する知見としてとても有効であるため,調査するべき課題である.

\par

そこで発表者はコンテナ型仮想化技術(Docker)におけるSATDの解消に関する調査を行っている.
具体的には,SATDの解消期間や解消方法を調査し,DockerにおけるSATDの解消パターンや特徴を,開発者を支援する知見として提供する予定である.

\par

本発表では,JavaのオープンソースプロジェクトにおけるSATDの解消や,Dockerにおけるバグ修正に関する論文の調査結果を報告する.
第2章では5つの大規模なオープンソースプロジェクトにおけるSATDの解消期間や解消の割合についての調査結果,第3章ではSATD解消時のソースコードやコミットメッセージへの影響についての調査結果,第4章ではDockerにおけるビルドの失敗率とその修正期間についての調査結果を報告し,第5章でまとめと今後の展望について述べる.
