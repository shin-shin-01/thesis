\section{まとめ}
第2章では5つの大規模なオープンソースプロジェクトにおけるSATDの解消期間や解消の割合についての調査結果,第3章ではSATD解消時のソースコードやコミットメッセージへの影響についての調査結果,第4章ではDockerにおけるビルドの失敗率とその修正期間についての調査結果を報告した.
Dcokerは比較的新しい技術であり,その開発手法が確立されていないことは,文献\cite{docker-failures}にて明確になっている.文献\cite{satd-real-removal},文献\cite{satd-real-removal}のようなSATDの削除に関する調査を行うことは,Dockerにおけるベストプラクティスな開発手法やSATD削除に関する知見の提供,開発支援を行うツールの作成につながると考えられる.
今後はこれらの調査手法や結果を参考に,DockerにおけるSATDの解消に関する調査を行っていく.
